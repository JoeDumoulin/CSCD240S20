\documentclass[10pt]{beamer}

\usetheme[progressbar=frametitle]{metropolis}

\usepackage[T1]{fontenc}
\usepackage{newunicodechar}
\usepackage[utf8]{inputenc}
%\usepackage{CJKutf8}
%\usepackage[russian]{babel}
%\usepackage{arabtex}

\usepackage{subcaption}
\usepackage{adjustbox}
\usepackage{booktabs}
\usepackage[scale=2]{ccicons}

% For pseudo codes
\usepackage{listings}
\usepackage{algorithm}
\usepackage[noend]{algpseudocode}
\makeatletter
\def\BState{\State\hskip-\ALG@thistlm}
\makeatother
%

\usepackage[cache=false]{minted}

\usepackage{multirow}
\usepackage[none]{hyphenat}
\usepackage{textcomp}
\usepackage{gensymb}
\sloppy 
%\usebackgroundtemplate


\usepackage{pgfplots}
\usepgfplotslibrary{dateplot}

\usepackage{xspace}
\newcommand{\themename}{\textbf{\textsc{metropolis}}\xspace}

\usepackage{tikz}
\usetikzlibrary{arrows,shapes,arrows.meta, positioning, quotes}

\setbeamercolor{background canvas}{bg=gray!20}

\title{Lecture 5}
\subtitle{C Pointers and Dynamic Memory}
\date{Mar 30, 2020}
\author{Joe Dumoulin}
\institute{Eastern Washington University}


% logo of ewu
\logo{\includegraphics[width=1cm]{../media/V_spot_EWUlogo.png}\hfill}
\newcommand{\nologo}{\setbeamertemplate{logo}{}} % command to set the logo to nothing
\newcommand{\congress}{Mar 30 - Jun 12 2020 CSCD 240-01}

\titlegraphic{\small\center C and Linux Programming
\\Eastern Washington University\\Computer Science\\March 30th – June 12th, 2020

\vspace{-15mm}\flushright\includegraphics[height=1.50cm]{../media/V_spot_EWUlogo.png}}

% footer
\makeatletter
\setbeamertemplate{footline}
{
  \leavevmode%
  \hbox{%

  \begin{beamercolorbox}[wd=.8\paperwidth,ht=2.25ex,dp=1ex,center]{institute in head/foot}%
    \usebeamerfont{abstract}%
    \congress
  \end{beamercolorbox}%

  \begin{beamercolorbox}[wd=.2\paperwidth,ht=2.25ex,dp=1ex,right]{institute in head/foot}%
    \usebeamerfont{abstract} 
    \insertframenumber{} / \inserttotalframenumber\hspace*{2ex} 
  \end{beamercolorbox}}%
  
}
\makeatother




% Document begin %%%%%%%%%%%%%%%%%%%%%%%%%%%%%%%%%%%%%%%%%%%%%%%%%%%%%%%%%%%%%%%%%%%%
\begin{document}

\maketitle
%%%%%%%%%%%%%%%%%%%%%%%%%%%%%%%%%%%%%%%%%%%%%%%%%%%%%%%%%%%%%%
\section{C Pointers and Dynamic Memory} 



%%%%%%%%%%%%%%%%%%%%%%
\begin{frame}[fragile]{The Story So Far}
Basic C syntax

\begin{itemize}

\item Basic C Syntax
\item Function Basics
\item File Handling

\end{itemize}

Next we will extend these ideas to understand pointers and dynamic memory management in C and Linux programming.

\end{frame}
%%%%%%%%%%%%%%%%%%%%%%
\begin{frame}[fragile]{A Closer Look at C Variables}
We use variables to store values while our program is running.  But what is happening under the covers?

One way to think about variables is as a reserved area of the computer's memory to store a value.  

Unlike files, variables only exist while a program is running.

A variable has four aspects:
\begin{itemize}
\item A \textbf{type},
\item A \textbf{name},
\item A \textbf{value} (or possibly it may be uninitialized)
\item An \textbf{address}  in memory.
\end{itemize}

\end{frame}
%%%%%%%%%%%%%%%%%%%%%%
\begin{frame}[fragile]{Definition of a pointer}
\textbf{A pointer is a variable whose value is the memory address of another variable.}

Consider the following code:
\fontsize{8pt}{8pt}
\selectfont
\begin{minted}{c}
int main () {

  int  var1;
  char var2[10];

  printf("Address of var1 variable: %p\n",  &var1  );
  printf("Address of var2 variable: %p\n",  &var2  );

  printf("var2 points to the beginning of a string. \
      \nIt also has a pointer value:%p\n", var2);

  return 0;
}
\end{minted}
This code will print out the addresses assigned to these different variables.  You can find the address of any variable by prefixing the variable name with "\&".



\end{frame}
%%%%%%%%%%%%%%%%%%%%%%
\begin{frame}[fragile]{Pointers and Addresses}

An example of the output from the code above on a computer with 64-bit addressing is:

\begin{minted}{bash}
Address of var1 variable:      0x7ffc56c156e8
Address of var2 variable:      0x7ffc56c156ee
var2 points to the beginning of a string.       
It also has a pointer value:   0x7ffc56c156ee
\end{minted}

Note that these addresses will be different each time the program runs!
   
\end{frame}
%%%%%%%%%%%%%%%%%%%%%%
\begin{frame}[fragile]{Pointers as Variables}

Here are some examples of pointer variables in C:

\begin{minted}{c}
int x = 0;    // an integer variable initialized to 0;
int* px = x;  // a pointer-to-integer variable 
              //   initialized to the address of x;

double pt1[3] = {5,7,6}; // a point in 3-D.
double* ppt1 = pt1;      // a pointer to the first element.
ppt1 += 1;               // now points to the second element.


// An array of strings (a 2-D array of characters)
char* strings[] = 
	{"An old take", "on a modern problem."};

\end{minted}

\end{frame}
%%%%%%%%%%%%%%%%%%%%%%
\begin{frame}[fragile]{Pointers and Functions}

C does not have reference variables.  Instead it uses pointer variables.  Pointer variables always include a type.  

Consider the problem of swapping a value between two integer variables.  we can do this with pointers.
\fontsize{8pt}{8pt}
\selectfont
\begin{minted}{c}
void swapints(int* p1, int* p2)
{
  // put the contents of memory location p1
  // in a local variable.
  int swp = *p1;

  // swap the contents of address p2 into 
  // address p1
  *p1 = *p2;

  // copy local variable value to address p2.
  *p2 = swp;
}
\end{minted}

\end{frame}
%%%%%%%%%%%%%%%%%%%%%%
\begin{frame}[fragile]{Using swapints}
The following main() function demonstrates how to use swapints:

\begin{minted}{c}
int main(void)
{
  int x1 = 50;
  int x2 = 63;
  printf("before swap, x1 = %d, x2 = %d.\n",
      x1, x2);

  swapints(&x1, &x2);

  printf("after swap, x1 = %d, x2 = %d.\n",
      x1,x2);
}

\end{minted}

A practical example of this would be sorting an array 
of integers.

\end{frame}
%%%%%%%%%%%%%%%%%%%%%%
\begin{frame}[fragile]{Pointers to Arrays}

A C array is really nothing more than a pointer to the first element of the array.  There is one crucial difference though.  The array variable's value is an address, but it is \textbf{constant}. It cannot change.  

Sometimes array variables are refered to as constant pointers.  We will have more to say about constant variables later.

we can get a non-constant pointer to the beginning of an array by assigning a pointer the value of the array variable like this:

\begin{minted}{c}

// the value of arr is the address of the first element.
int arr[] = {1,2}; // arr is constant.  

// the value of p is also the address of the first element of arr
int* p = arr;

\end{minted}

\end{frame}
%%%%%%%%%%%%%%%%%%%%%%
\begin{frame}[fragile]{Example: Copying Strings}
Since strings are arrays in C, we can use them to illustrate some of the ways you can work with pointers in arrays.  For example, consider copying a string into a buffer.  The code might look like the below function;

\fontsize{8pt}{8pt}
\selectfont
\begin{minted}{c}
#define BUF_SZ 200

int main(void)
{
  char* str = "Be like a headland of rock on \
which the waves break incessantly but it \
stands fast and around it the seething \
of the waters sinks to rest.";

  char buffer[BUF_SZ] = "";
  int i;

  // copy str to buffer, then print buffer
  for (i=0; i<strlen(str); i++) {
    buffer[i] = str[i];
  }
  printf("%s\n", buffer);

  return EXIT_SUCCESS;
}
\end{minted}

\end{frame}
%%%%%%%%%%%%%%%%%%%%%%
\begin{frame}[fragile]{Example: Copying Strings}

The above works, but C offers some alternatives that involve pointers.  Below is a function for the copy:

\begin{minted}{c}
// copy the contents of p1 to p2.  
// Assume '\0' ending strings.
void stringcpy(char* p1, char* p2)
{
  while(*p1) {  // while p1 != 0 (note: 0 == '\0')
    *p2 = *p1;
    p1++;
    p2++;
  }
}
  // in main we call stringcpy like this:
  stringcpy(str, buffer);
  ...
\end{minted}

\end{frame}
%%%%%%%%%%%%%%%%%%%%%%
\begin{frame}[fragile]{Example: Copying Strings}

We can use a  principle of C programming to make stringcpy even more compact.  First note that the assignment operator returns the value of the assignment.  we can then move the assignment into the while predicate:

\begin{minted}{c}
void stringcpy(char* p1, char* p2)
{
  while(*p2 = *p1) { // while *p1 != 0,
    p1++;
    p2++;
  }
}
\end{minted}

\end{frame}
%%%%%%%%%%%%%%%%%%%%%%
\begin{frame}[fragile]{Example: Copying Strings}

Next we use the precedence of operators to further condense this expression.  Note that ++ on the right is applied to the variable \textit{after} *.  that means we can get the value from *p1 and assign it to *p2 before we increment p1 and p2 with this line:

\begin{minted}{c}
*p2++ = *p1++;
\end{minted}
With that, we can change the function to:
\begin{minted}{c}
void stringcpy(char* p1, char* p2)
{
  while(*p2++ = *p1++)  // while *p1 != 0,
    ;
}
\end{minted}

\end{frame}
%%%%%%%%%%%%%%%%%%%%%%
\begin{frame}[fragile]{Pointers and Dynamic Memory}

Most languages have the ability to create new objects.  The basic C language has no ability to do this.  But we can use the C library call \textbf{malloc} to help us accomplish the same thing.

Creating objects has two steps.  
\begin{itemize}
\item Reserve meomory for the object.
\item Initialize the obect.
\end{itemize}

Most languages combine these steps into a single operation called "new".  In C, we have to do each of these separately.

\end{frame}
%%%%%%%%%%%%%%%%%%%%%%
\begin{frame}[fragile]{Managing Dynamic Memory}

Most languages provide a means to clean up memory.  Usually they use a system called a \textbf{Garbage Collector}. A Garbage Collector (GC) is part of a language system and it keeps track of all allocated objects until they are no loger referenced.  Once objects are no longer referenced, the GC removes them from memory.  

C has no GC.  You must manually remove objects that were allocated with \textbf{malloc}.  The C library provides the function \textbf{free} to accomplish this.

\end{frame}
%%%%%%%%%%%%%%%%%%%%%%
\begin{frame}[fragile]{Managing Dynamic Memory - An Example}

Here is an example program that allocates a string instead of using a static buffer like we did before.
\fontsize{8pt}{8pt}
\selectfont
\begin{minted}{c}
int main(void)
{
  char* quote = "Be like a headland of rock on \
which the waves break incessantly but it \
stands fast and around it the seething \
of the waters sinks to rest.";
  printf("original string:\n\n%s\n\n", quote);

  char* copyquote; // a pointer to nowhere... yet.

  // malloc gives us a memory location for our
  // copy of the quote.
  copyquote = (char*) malloc(strlen(quote));

  // now copy the quote into the new space
  stringcpy(quote, copyquote);

  printf("copied string:\n\n%s\n\n", copyquote);
  
  // free the allocated memory when we are done.
  free(copyquote);
}
\end{minted}

\end{frame}
%%%%%%%%%%%%%%%%%%%%%%
\begin{frame}[fragile]{A Closer Look at malloc}
\textbf{malloc} has some important new aspects.  here's the code again:

\begin{minted}{c}
...
  copyquote = (char*) malloc(strlen(quote));
...
\end{minted}

\begin{itemize}
\item \textbf{copyquote} is an uninitialized pointer variable.
\item \textbf{(char*)} is a \textbf{cast} from the return value from \textbf{malloc}. More about this below.
\item \textbf{strlen(quote)} gets the proper length of the string (including the terminator '\textbackslash 0').  This is the number of bytes to allocate.  \textbf{malloc} always takes the number of \textbf{bytes} as a parameter.
\end{itemize}

\end{frame}
%%%%%%%%%%%%%%%%%%%%%%
\begin{frame}[fragile]{\textbf{malloc} return type: \textbf{void *}}

\textbf{malloc} uses a special type of pointer for its return value: \textbf{void*}. If \textbf{malloc} is successful and allocates the memory asked for, then the pointer returned points to the new memory.

The pointer has type \textbf{void *} which must be cast to the type that you want.  We will see other examples of \textbf{malloc} in the next lecture.

\end{frame}
%%%%%%%%%%%%%%%%%%%%%%
\begin{frame}[fragile]{The Next Assignment}

In the next assignment, you will change the static buffer from a file copying program to allocate memory dynamically., and then free the buffer memory back to the heap.

\end{frame}

\end{document}